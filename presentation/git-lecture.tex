% Created 2017-11-10 Пт. 11:50
% -*- mode: LaTeX; coding: utf-8; -*-

\documentclass[russian,utf8]{beamer} 
\usepackage[T2A]{fontenc}
\usepackage[utf8]{inputenc}
\usepackage[english,russian]{babel}
\usepackage{graphicx}
\usepackage{beamerthemesplit}
\usetheme{Berkeley}
\usecolortheme{dolphin}
\setbeamertemplate{footline}[text line]{}
% \graphicspath{{../image/}}

\title{Основы Git}
\author{Sergey Popov}
\date{2017-11-10 Пт.}
\hypersetup{
  pdfkeywords={},
  pdfsubject={},
  pdfcreator={Emacs Org-mode version 7.9.3f}}

%\usetheme{PaloAlto}
%\usetheme{Singapore}
%\usetheme{Madrid}
%\useoutertheme{tree}
%\usecolortheme{rose}
%\useinnertheme{rectangles}

%%% Local Variables:
%%% mode: latex
%%% TeX-master: "main"
%%% End:


\begin{document}

\maketitle

\begin{frame}
\frametitle{Outline}
\setcounter{tocdepth}{3}
\tableofcontents
\end{frame}
\section{Введение}
\label{sec-1}

\href{https://git-scm.com/book/ru/v2/%D0%92%D0%B2%D0%B5%D0%B4%D0%B5%D0%BD%D0%B8%D0%B5-%D0%9E-%D1%81%D0%B8%D1%81%D1%82%D0%B5%D0%BC%D0%B5-%D0%BA%D0%BE%D0%BD%D1%82%D1%80%D0%BE%D0%BB%D1%8F-%D0%B2%D0%B5%D1%80%D1%81%D0%B8%D0%B9}{https://git-scm.com/book/ru/v2/\%D0\%92\%D0\%B2\%D0\%B5\%D0\%B4\%D0\%B5\%D0\%BD\%D0\%B8\%D0\%B5-\%D0\%9E-\%D1\%81\%D0\%B8\%D1\%81\%D1\%82\%D0\%B5\%D0\%BC\%D0\%B5-\%D0\%BA\%D0\%BE\%D0\%BD\%D1\%82\%D1\%80\%D0\%BE\%D0\%BB\%D1\%8F-\%D0\%B2\%D0\%B5\%D1\%80\%D1\%81\%D0\%B8\%D0\%B9}
Систему упраления версиями (VCS - Version Control System) бывают:
\begin{enumerate}
\item Ручные - copy/rename и т.д.
\item Локальные - RCS.
\item Централизованными - CSV, Subversion и др.
\item Распределёнными - Git, Mercurial, Bazaar и др.
\end{enumerate}

Основные различия на примере Git и Subversion:

\begin{center}
\begin{tabular}{rll}
    &  svn                              &  git                                   \\
\hline
 1  &  клиент-сервеная модель           &  распределённая модель                 \\
 2  &  1 репозитарий для всех проектов  &  для каждого проекта свой репозитарий  \\
\end{tabular}
\end{center}
\begin{frame}
\frametitle{Структура проекта}
\label{sec-1-1}

В корне проекта всегда должна быть директория .git
\end{frame}
\begin{frame}
\frametitle{Внутреннее устройство, как работает.}
\label{sec-1-2}

? Наверняка кому интересно уже вкурсе.
\href{https://git-scm.com/book/ru/v2/%D0%92%D0%B2%D0%B5%D0%B4%D0%B5%D0%BD%D0%B8%D0%B5-%D0%9E%D1%81%D0%BD%D0%BE%D0%B2%D1%8B-Git}{https://git-scm.com/book/ru/v2/\%D0\%92\%D0\%B2\%D0\%B5\%D0\%B4\%D0\%B5\%D0\%BD\%D0\%B8\%D0\%B5-\%D0\%9E\%D1\%81\%D0\%BD\%D0\%BE\%D0\%B2\%D1\%8B-Git}
\href{https://git-scm.com/book/ru/v1/%D0%92%D0%B5%D1%82%D0%B2%D0%BB%D0%B5%D0%BD%D0%B8%D0%B5-%D0%B2-Git-%D0%A7%D1%82%D0%BE-%D1%82%D0%B0%D0%BA%D0%BE%D0%B5-%D0%B2%D0%B5%D1%82%D0%BA%D0%B0%3F}{https://git-scm.com/book/ru/v1/\%D0\%92\%D0\%B5\%D1\%82\%D0\%B2\%D0\%BB\%D0\%B5\%D0\%BD\%D0\%B8\%D0\%B5-\%D0\%B2-Git-\%D0\%A7\%D1\%82\%D0\%BE-\%D1\%82\%D0\%B0\%D0\%BA\%D0\%BE\%D0\%B5-\%D0\%B2\%D0\%B5\%D1\%82\%D0\%BA\%D0\%B0\%3F}
\end{frame}
\section{Настройка}
\label{sec-2}
\begin{frame}
\frametitle{gitconfig (команда git config)}
\label{sec-2-1}

\begin{itemize}
\item Note taken on \textit{2017-10-22 Вс 21:22} \\
Посмотреть \href{https://github.com/BenBergman/.gitconfig.d}{https://github.com/BenBergman/.gitconfig.d}
\end{itemize}
Команда `git config`. 
\begin{itemize}

\item core (autocrlf, safecrlf)\\
\label{sec-2-1-1}%
\begin{center}
\begin{tabular}{ll}
 windows                        &  unix                            \\
\hline
 git config core.autocrlf true  &  git config core.autocrlf input  \\
\end{tabular}
\end{center}


В итоге, для windows будет работать автоматическое преобразование
crlf->lf при выгрузке кода. А для linux будет произведено
конвертирование crlf->lf только при коммите.

\item alias (+ bash$_{\mathrm{aliases}}$) + github
\label{sec-2-1-2}%

\item merge/mergetool\\
\label{sec-2-1-3}%
При слиянии веток, могут возникнуть конфликты. По умолчанию, конфликты
нужно решить вручную. Но можно использовать специализированные
утилиты.

`git help merge`

\item git config -l - вывод всех настроек
\label{sec-2-1-4}%

\item Включение по условию includeIf (начиная с версии 2.13)\\
\label{sec-2-1-5}%
\href{https://stackoverflow.com/questions/4220416/can-i-specify-multiple-users-for-myself-in-gitconfig}{https://stackoverflow.com/questions/4220416/can-i-specify-multiple-users-for-myself-in-gitconfig}
\href{https://git-scm.com/docs/git-config#_conditional_includes}{https://git-scm.com/docs/git-config\#\_conditional\_includes}
\end{itemize} % ends low level
\end{frame}
\begin{frame}
\frametitle{gitignore}
\label{sec-2-2}

\href{https://github.com/github/gitignore}{https://github.com/github/gitignore} - репозитарий файлов gitignore для
любых языков и платформ
\end{frame}
\begin{frame}
\frametitle{gitattributes}
\label{sec-2-3}

\href{https://git-scm.com/book/ru/v1/%D0%9D%D0%B0%D1%81%D1%82%D1%80%D0%BE%D0%B9%D0%BA%D0%B0-Git-Git-%D0%B0%D1%82%D1%80%D0%B8%D0%B1%D1%83%D1%82%D1%8B}{https://git-scm.com/book/ru/v1/\%D0\%9D\%D0\%B0\%D1\%81\%D1\%82\%D1\%80\%D0\%BE\%D0\%B9\%D0\%BA\%D0\%B0-Git-Git-\%D0\%B0\%D1\%82\%D1\%80\%D0\%B8\%D0\%B1\%D1\%83\%D1\%82\%D1\%8B}
\begin{itemize}

\item EOL
\label{sec-2-3-1}%
\begin{itemize}
\item Note taken on \textit{2017-11-06 Пн 10:57} \\
Файл .gitattributes позволяет настроить переносы строк для любых
  файлов, чтобы у всех они были одинаковые, и не было беспорядка в
  дифах. Также если в вашем GitHub репозитории некорректно определяется
  язык, то часть файлов также можно проигнорировать с помощью аттрибута
  linguist-vendored. Вот пример файла .gitattributes с \n разрывами
  строк и игнорируемыми файлами по фильтру **/examples/*:
  
  \textbf{.} eol=lf
  **/examples/* linguist-vendored
\end{itemize}

\item diff для бинарный файлов\\
\label{sec-2-3-2}%
В .gitconfig 
[diff ``wordx'']
     textconv = docx2txt <

В .gitattributes
*.docx diff=wordx
*.docx difftool=wordx


\item Стратегии слияния\\
\label{sec-2-3-3}%
\href{https://git-scm.com/book/ru/v1/%D0%9D%D0%B0%D1%81%D1%82%D1%80%D0%BE%D0%B9%D0%BA%D0%B0-Git-Git-%D0%B0%D1%82%D1%80%D0%B8%D0%B1%D1%83%D1%82%D1%8B#%D0%A1%D1%82%D1%80%D0%B0%D1%82%D0%B5%D0%B3%D0%B8%D0%B8-%D1%81%D0%BB%D0%B8%D1%8F%D0%BD%D0%B8%D1%8F}{https://git-scm.com/book/ru/v1/\%D0\%9D\%D0\%B0\%D1\%81\%D1\%82\%D1\%80\%D0\%BE\%D0\%B9\%D0\%BA\%D0\%B0-Git-Git-\%D0\%B0\%D1\%82\%D1\%80\%D0\%B8\%D0\%B1\%D1\%83\%D1\%82\%D1\%8B\#\%D0\%A1\%D1\%82\%D1\%80\%D0\%B0\%D1\%82\%D0\%B5\%D0\%B3\%D0\%B8\%D0\%B8-\%D1\%81\%D0\%BB\%D0\%B8\%D1\%8F\%D0\%BD\%D0\%B8\%D1\%8F}
database.xml merge=ours
\end{itemize} % ends low level
\end{frame}
\begin{frame}
\frametitle{\href{http://editorconfig.org/}{http://editorconfig.org/}}
\label{sec-2-4}
\end{frame}
\section{Команды}
\label{sec-3}

Для отображения результатов выполнения команд help, log, diff
используется программа, заданная в .gitconfig, например: 

git config --global core.pager ``less -XF''
\begin{frame}
\frametitle{git help}
\label{sec-3-1}

Самая нужная команда 
\end{frame}
\begin{frame}
\frametitle{git init}
\label{sec-3-2}

git init project$_{\mathrm{name}}$
\end{frame}
\begin{frame}
\frametitle{git diff}
\label{sec-3-3}

Жизненный цикл состояний файлов
\href{https://git-scm.com/book/ru/v1/%D0%9E%D1%81%D0%BD%D0%BE%D0%B2%D1%8B-Git-%D0%97%D0%B0%D0%BF%D0%B8%D1%81%D1%8C-%D0%B8%D0%B7%D0%BC%D0%B5%D0%BD%D0%B5%D0%BD%D0%B8%D0%B9-%D0%B2-%D1%80%D0%B5%D0%BF%D0%BE%D0%B7%D0%B8%D1%82%D0%BE%D1%80%D0%B8%D0%B9}{https://git-scm.com/book/ru/v1/\%D0\%9E\%D1\%81\%D0\%BD\%D0\%BE\%D0\%B2\%D1\%8B-Git-\%D0\%97\%D0\%B0\%D0\%BF\%D0\%B8\%D1\%81\%D1\%8C-\%D0\%B8\%D0\%B7\%D0\%BC\%D0\%B5\%D0\%BD\%D0\%B5\%D0\%BD\%D0\%B8\%D0\%B9-\%D0\%B2-\%D1\%80\%D0\%B5\%D0\%BF\%D0\%BE\%D0\%B7\%D0\%B8\%D1\%82\%D0\%BE\%D1\%80\%D0\%B8\%D0\%B9}
\end{frame}
\begin{frame}
\frametitle{git log}
\label{sec-3-4}

\href{https://git-scm.com/book/ru/v1/%D0%98%D0%BD%D1%81%D1%82%D1%80%D1%83%D0%BC%D0%B5%D0%BD%D1%82%D1%8B-Git-%D0%92%D1%8B%D0%B1%D0%BE%D1%80-%D1%80%D0%B5%D0%B2%D0%B8%D0%B7%D0%B8%D0%B8}{https://git-scm.com/book/ru/v1/\%D0\%98\%D0\%BD\%D1\%81\%D1\%82\%D1\%80\%D1\%83\%D0\%BC\%D0\%B5\%D0\%BD\%D1\%82\%D1\%8B-Git-\%D0\%92\%D1\%8B\%D0\%B1\%D0\%BE\%D1\%80-\%D1\%80\%D0\%B5\%D0\%B2\%D0\%B8\%D0\%B7\%D0\%B8\%D0\%B8}
--decorate
--stat
--oneline
--graph
\end{frame}
\begin{frame}
\frametitle{commit}
\label{sec-3-5}
\end{frame}
\begin{frame}
\frametitle{branch}
\label{sec-3-6}

Команда умеет выводить списко веток и удалять их.
Переходы между ветками осуществляются через checkout.
\end{frame}
\begin{frame}
\frametitle{merge}
\label{sec-3-7}
\begin{itemize}

\item 3-way
\label{sec-3-7-1}%

\item Конфликты
\label{sec-3-7-2}%
\end{itemize} % ends low level
\end{frame}
\begin{frame}
\frametitle{rebase}
\label{sec-3-8}

Запомнить! Изменять историю коммитов следует только в локальном репозитории.
\begin{itemize}

\item interactive\\
\label{sec-3-8-1}%
\href{https://git-scm.com/book/ru/v1/%D0%98%D0%BD%D1%81%D1%82%D1%80%D1%83%D0%BC%D0%B5%D0%BD%D1%82%D1%8B-Git-%D0%9F%D0%B5%D1%80%D0%B5%D0%B7%D0%B0%D0%BF%D0%B8%D1%81%D1%8C-%D0%B8%D1%81%D1%82%D0%BE%D1%80%D0%B8%D0%B8}{https://git-scm.com/book/ru/v1/\%D0\%98\%D0\%BD\%D1\%81\%D1\%82\%D1\%80\%D1\%83\%D0\%BC\%D0\%B5\%D0\%BD\%D1\%82\%D1\%8B-Git-\%D0\%9F\%D0\%B5\%D1\%80\%D0\%B5\%D0\%B7\%D0\%B0\%D0\%BF\%D0\%B8\%D1\%81\%D1\%8C-\%D0\%B8\%D1\%81\%D1\%82\%D0\%BE\%D1\%80\%D0\%B8\%D0\%B8}
\end{itemize} % ends low level
\end{frame}
\begin{frame}
\frametitle{cherry-pick}
\label{sec-3-9}
\end{frame}
\begin{frame}
\frametitle{bisect}
\label{sec-3-10}

\href{https://git-scm.com/book/ru/v1/%D0%98%D0%BD%D1%81%D1%82%D1%80%D1%83%D0%BC%D0%B5%D0%BD%D1%82%D1%8B-Git-%D0%9E%D1%82%D0%BB%D0%B0%D0%B4%D0%BA%D0%B0-%D1%81-%D0%BF%D0%BE%D0%BC%D0%BE%D1%89%D1%8C%D1%8E-Git#%D0%91%D0%B8%D0%BD%D0%B0%D1%80%D0%BD%D1%8B%D0%B9-%D0%BF%D0%BE%D0%B8%D1%81%D0%BA}{https://git-scm.com/book/ru/v1/\%D0\%98\%D0\%BD\%D1\%81\%D1\%82\%D1\%80\%D1\%83\%D0\%BC\%D0\%B5\%D0\%BD\%D1\%82\%D1\%8B-Git-\%D0\%9E\%D1\%82\%D0\%BB\%D0\%B0\%D0\%B4\%D0\%BA\%D0\%B0-\%D1\%81-\%D0\%BF\%D0\%BE\%D0\%BC\%D0\%BE\%D1\%89\%D1\%8C\%D1\%8E-Git\#\%D0\%91\%D0\%B8\%D0\%BD\%D0\%B0\%D1\%80\%D0\%BD\%D1\%8B\%D0\%B9-\%D0\%BF\%D0\%BE\%D0\%B8\%D1\%81\%D0\%BA}
\end{frame}
\begin{frame}
\frametitle{stash}
\label{sec-3-11}

\href{https://git-scm.com/book/ru/v1/%D0%98%D0%BD%D1%81%D1%82%D1%80%D1%83%D0%BC%D0%B5%D0%BD%D1%82%D1%8B-Git-%D0%9F%D1%80%D1%8F%D1%82%D0%B0%D0%BD%D1%8C%D0%B5}{https://git-scm.com/book/ru/v1/\%D0\%98\%D0\%BD\%D1\%81\%D1\%82\%D1\%80\%D1\%83\%D0\%BC\%D0\%B5\%D0\%BD\%D1\%82\%D1\%8B-Git-\%D0\%9F\%D1\%80\%D1\%8F\%D1\%82\%D0\%B0\%D0\%BD\%D1\%8C\%D0\%B5}
\end{frame}
\begin{frame}
\frametitle{git blame}
\label{sec-3-12}

\href{https://github.com/git/git/commit/d69360c6b17d1693a60b9f723a3ef5129a62c2e5?diff=unified}{https://github.com/git/git/commit/d69360c6b17d1693a60b9f723a3ef5129a62c2e5?diff=unified}
\end{frame}
\begin{frame}
\frametitle{git grep}
\label{sec-3-13}

Поиск по истории коммитов и файлам.
\end{frame}
\begin{frame}
\frametitle{pull request}
\label{sec-3-14}
\end{frame}
\section{Hooks}
\label{sec-4}

? Обычным пользователям не нужно.
\href{https://git-scm.com/book/ru/v1/%D0%9D%D0%B0%D1%81%D1%82%D1%80%D0%BE%D0%B9%D0%BA%D0%B0-Git-%D0%9F%D0%B5%D1%80%D0%B5%D1%85%D0%B2%D0%B0%D1%82%D1%87%D0%B8%D0%BA%D0%B8-%D0%B2-Git}{https://git-scm.com/book/ru/v1/\%D0\%9D\%D0\%B0\%D1\%81\%D1\%82\%D1\%80\%D0\%BE\%D0\%B9\%D0\%BA\%D0\%B0-Git-\%D0\%9F\%D0\%B5\%D1\%80\%D0\%B5\%D1\%85\%D0\%B2\%D0\%B0\%D1\%82\%D1\%87\%D0\%B8\%D0\%BA\%D0\%B8-\%D0\%B2-Git}
\section{Типичные сценарии работы}
\label{sec-5}

\href{https://www.atlassian.com/git/tutorials/comparing-workflows}{https://www.atlassian.com/git/tutorials/comparing-workflows}
\begin{frame}
\frametitle{Работа с длинными ветками}
\label{sec-5-1}

\href{https://git-scm.com/book/ru/v1/%D0%92%D0%B5%D1%82%D0%B2%D0%BB%D0%B5%D0%BD%D0%B8%D0%B5-%D0%B2-Git-%D0%9F%D1%80%D0%B8%D1%91%D0%BC%D1%8B-%D1%80%D0%B0%D0%B1%D0%BE%D1%82%D1%8B-%D1%81-%D0%B2%D0%B5%D1%82%D0%BA%D0%B0%D0%BC%D0%B8}{https://git-scm.com/book/ru/v1/\%D0\%92\%D0\%B5\%D1\%82\%D0\%B2\%D0\%BB\%D0\%B5\%D0\%BD\%D0\%B8\%D0\%B5-\%D0\%B2-Git-\%D0\%9F\%D1\%80\%D0\%B8\%D1\%91\%D0\%BC\%D1\%8B-\%D1\%80\%D0\%B0\%D0\%B1\%D0\%BE\%D1\%82\%D1\%8B-\%D1\%81-\%D0\%B2\%D0\%B5\%D1\%82\%D0\%BA\%D0\%B0\%D0\%BC\%D0\%B8}
\end{frame}
\begin{frame}
\frametitle{GitFlow}
\label{sec-5-2}

\href{https://git-scm.com/book/ru/v1/%D0%9D%D0%B0%D1%81%D1%82%D1%80%D0%BE%D0%B9%D0%BA%D0%B0-Git-%D0%9F%D0%B5%D1%80%D0%B5%D1%85%D0%B2%D0%B0%D1%82%D1%87%D0%B8%D0%BA%D0%B8-%D0%B2-Git}{https://git-scm.com/book/ru/v1/\%D0\%9D\%D0\%B0\%D1\%81\%D1\%82\%D1\%80\%D0\%BE\%D0\%B9\%D0\%BA\%D0\%B0-Git-\%D0\%9F\%D0\%B5\%D1\%80\%D0\%B5\%D1\%85\%D0\%B2\%D0\%B0\%D1\%82\%D1\%87\%D0\%B8\%D0\%BA\%D0\%B8-\%D0\%B2-Git}
\end{frame}
\section{Полезное}
\label{sec-6}
\begin{frame}
\frametitle{git help}
\label{sec-6-1}
\end{frame}
\begin{frame}
\frametitle{git-extras}
\label{sec-6-2}
\begin{itemize}

\item bug
\label{sec-6-2-1}%

\item feature
\label{sec-6-2-2}%

\item release
\label{sec-6-2-3}%

\item refactor
\label{sec-6-2-4}%

\item summary
\label{sec-6-2-5}%

\item info
\label{sec-6-2-6}%

\item changelog
\label{sec-6-2-7}%
\end{itemize} % ends low level
\end{frame}
\begin{frame}
\frametitle{tig}
\label{sec-6-3}
\end{frame}
\begin{frame}
\frametitle{gitk}
\label{sec-6-4}
\end{frame}

\end{document}
